\begin{figure}[H]
\centering
\begin{tikzpicture}[
    node distance=0.1cm,
    layer/.style={
        draw,
        rectangle,
        thick,
        minimum width=8cm,
        minimum height=1.3cm,
        text centered,
        font=\sffamily
    },
    pdu/.style={
        font=\sffamily\small\itshape,
        text=gray
    },
    desc/.style={
        font=\sffamily\small,
        align=left
    },
    arrow/.style={
        -latex,
        ultra thick
    }
]

\node (L7) [layer, fill=red!20]                 {\textbf{7. Прикладной уровень}};
\node (L6) [layer, fill=orange!20, below=of L7] {\textbf{6. Уровень представления}};
\node (L5) [layer, fill=yellow!20, below=of L6] {\textbf{5. Сеансовый уровень}};
\node (L4) [layer, fill=green!20, below=of L5]  {\textbf{4. Транспортный уровень}};
\node (L3) [layer, fill=cyan!20, below=of L4]   {\textbf{3. Сетевой уровень}};
\node (L2) [layer, fill=blue!20, below=of L3]   {\textbf{2. Канальный уровень}};
\node (L1) [layer, fill=violet!20, below=of L2] {\textbf{1. Физический уровень}};

%\node[desc, right=0.5cm of L7] {Пользовательские данные \\ \textit{(HTTP, FTP, SMTP, DNS)}};
%\node[desc, right=0.5cm of L6] {Представление и кодирование данных \\ \textit{(SSL/TLS, ASCII, JPEG)}};
%\node[desc, right=0.5cm of L5] {Управление сеансом связи \\ \textit{(NetBIOS, RPC)}};
%\node[desc, right=0.5cm of L4] {\textbf{Сегменты / Дейтаграммы} \\ \textit{(TCP, UDP)}};
%\node[desc, right=0.5cm of L3] {\textbf{Пакеты} \\ \textit{(IP, ICMP, IGMP)}};
%\node[desc, right=0.5cm of L2] {\textbf{Кадры (фреймы)} \\ \textit{(Ethernet, Wi-Fi, MAC-адрес)}};
%\node[desc, right=0.5cm of L1] {\textbf{Биты} \\ \textit{(Кабель, оптоволокно, радиосигнал)}};


\end{tikzpicture}
\caption{Схема семиуровневой модели OSI}
\label{fig:osi_model}
\end{figure}
