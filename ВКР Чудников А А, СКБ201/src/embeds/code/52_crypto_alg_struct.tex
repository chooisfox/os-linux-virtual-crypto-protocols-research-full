\begin{minted}[tabsize=4]{c}
struct crypto_alg {
    struct list_head cra_list;
    struct list_head cra_users;

    u32 cra_flags; /* Набор флагов, описывающих алгоритм */

    unsigned int cra_blocksize; /* Размер блока шифра в байтах */
    unsigned int cra_ctxsize;   /* Размер структуры контекста */
    unsigned int cra_alignmask; /* Маска выравнивания входных
                                   и выходных данных */

    int cra_priority; /* Приоритет реализации */

    refcount_t cra_refcnt;

    char cra_name[CRYPTO_MAX_ALG_NAME];        /* Имя алгоритма */
    char cra_driver_name[CRYPTO_MAX_ALG_NAME]; /* Имя драйвера */

    const struct crypto_type *cra_type;  /* Тип криптопреобразования */
    /* Указатель на структуру crypto_type, реализующую
     * общие функции обратного вызова.
     *
     * Доступные варианты:
     * - &crypto_blkcipher_type
     * - &crypto_ablkcipher_type
     * - &crypto_ahash_type
     * - &crypto_rng_type
     *
     * Оставляется пустым для шифрования блока (cipher),
     * сжатия (compress) и блокирующего хэширования (shash).
     */

    union {
        struct cipher_alg cipher;
        struct compress_alg compress;
    } cra_u;

    int (*cra_init)(struct crypto_tfm *tfm);
    void (*cra_exit)(struct crypto_tfm *tfm);
    void (*cra_destroy)(struct crypto_alg *alg);

    struct module *cra_module;
};
\end{minted}
