\section{Программная реализация и тестирование}

Разработанный автором работы программный комплекс, реализующий виртуальный криптографический сетевой интерфейс, получил название \texttt{WireGost}.
Ключевым архитектурным принципом разработки стала максимальная интеграция с существующей инфраструктурой ядра \texttt{Linux} и использование стандартных механизмов операционной системы.\\

С целью упрощения разработки, для всех используемых криптографических модулей ядра были написаны <<функции-обёртки>>, позволяющие намного проще вносить изменения в реализацию модулей, без необходимости внесения изменений в основную часть кода модуля \texttt{WireGost}. ~\SeeApp{app:wiregost}

\subsection{Реализация криптографических примитивов ГОСТ}

\subsubsection{Алгоритм хэширования ГОСТ Р 34.11-2012}

Анализ исходного кода ядра \texttt{Linux} (для версии 5.10 и выше) показал, что алгоритм ГОСТ Р 34.11-2012 (<<Стрибог>>) не требует разработки, так как он уже является частью \texttt{Crypto API}. Несмотря на то, что изменение модуля не требовалось, автором было решено всё равно использовать промежуточный слой абстракции, упомянутый ранее, с целью того, чтобы держать кодовую базу в более однородном формате. Этот слой абстракции был расположен в файле \texttt{gost/gost\_streebog.c}.

\nlparagraph{Инициализация и управление контекстом}

Благодаря использованию промежуточного слоя абстракции, удалось незначительно оптимизировать работу модуля путем инициализации дескриптора алгоритма (\texttt{tfm} — transformation object) единожды при загрузке модуля, а не при каждом вычислении хэша.\\

Функция инициализации \texttt{gost\_streebog\_init\_module} выполняет следующие критические операции:
\begin{enumerate}
    \item запрос алгоритма \texttt{"streebog256"} через вызов \texttt{crypto\_alloc\_shash};
    \item контроль корректности выделения ресурсов;
    \item проверка размера контекста хэширования. Поскольку структура состояния \texttt{struct gost\_streebog\_state} размещается на стеке ядра, необходимо гарантировать, что её размер не превышает зарезервированный буфер \texttt{GOST\_STREEBOG\_CTX\_SIZE}, чтобы избежать переполнения стека.
\end{enumerate}

\nlparagraph{Программный интерфейс}
Реализован набор <<функций-оберток>>, инкапсулирующих вызовы Crypto API:
\begin{itemize}
    \item \texttt{gost\_streebog256\_init}: связывание контекста с глобальным дескриптором и сброс состояния;
    \item \texttt{gost\_streebog256\_update}: подача данных на вход функции сжатия;
    \item \texttt{gost\_streebog256\_final}: финализация вычислений и формирование дайджеста.
\end{itemize}

\begin{minted}[tabsize=4]{c}
#include "gost_streebog.h"
#include <linux/err.h>
#include <linux/module.h>
#include <linux/slab.h>
#include <linux/string.h>

static struct crypto_shash *streebog_tfm;

int gost_streebog_init_module(void)
{
    streebog_tfm = crypto_alloc_shash("streebog256", 0, 0);
    if (IS_ERR(streebog_tfm)) {
        pr_err("WireGost: failed to allocate streebog256: %ld\n", PTR_ERR(streebog_tfm));
        return PTR_ERR(streebog_tfm);
    }
    if (crypto_shash_descsize(streebog_tfm) > GOST_STREEBOG_CTX_SIZE) {
        crypto_free_shash(streebog_tfm);
        return -ENOMEM;
    }
    return 0;
}

void gost_streebog_cleanup_module(void)
{
    if (streebog_tfm) {
        crypto_free_shash(streebog_tfm);
        streebog_tfm = NULL;
    }
}

void gost_streebog256_init(struct gost_streebog_state *ctx)
{
    if (likely(streebog_tfm)) {
        ctx->desc.tfm = streebog_tfm;
        crypto_shash_init(&ctx->desc);
    }
}

void gost_streebog256_update(struct gost_streebog_state *ctx, const u8 *data, size_t len)
{
    crypto_shash_update(&ctx->desc, data, len);
}

void gost_streebog256_final(struct gost_streebog_state *ctx, u8 *hash)
{
    crypto_shash_final(&ctx->desc, hash);
}

void gost_streebog256(u8 *out, const u8 *in, size_t len)
{
    struct gost_streebog_state ctx;
    gost_streebog256_init(&ctx);
    gost_streebog256_update(&ctx, in, len);
    gost_streebog256_final(&ctx, out);
    memzero_explicit(&ctx, sizeof(ctx));
}
\end{minted}


\subsubsection{Конструкция HMAC на базе ГОСТ Р 34.11-2012}

Оригинальный протокол \texttt{WireGuard} использует хэш-функцию \texttt{BLAKE2s}, поддерживающую режим <<хэширования с ключом>>. Алгоритм <<Стрибог>>, реализованный в ядре, является <<чистой>> хэш-функцией и не поддерживает инициализацию ключом. \\

Для обеспечения аутентификации сообщений и реализации функций формирования ключа (\texttt{KDF}) была программно реализована схема \texttt{HMAC} (\texttt{Hash-based Message Authentication Code}) в соответствии с \texttt{RFC 2104}~\cite{rfc2104} и рекомендациями \texttt{RFC 7836}~\cite{rfc7836}.\\

HMAC-Стрибог был реализован в файле \texttt{gost/gost\_hmac.c}, сам алгоритм при этом базируется на формуле:
$$
    \text{HMAC}(K, m) = H((K' \oplus opad) \parallel H((K' \oplus ipad) \parallel m))
$$

Где $K'$ — нормализованный ключ, $ipad$ и $opad$ — константы внутреннего и внешнего дополнения соответственно.

\nlparagraph{Особенности реализации}

Ключевым требованием к реализации в пространстве ядра является безопасная работа с памятью. Функция \texttt{gost\_hmac256} обеспечивает принудительную очистку всех буферов, содержащих производные ключа и промежуточные хэши, с помощью вызова \texttt{memzero\_explicit}, исключая утечку ключевого материала.

\begin{minted}[tabsize=4]{c}
void gost_hmac256(u8 *out, const u8 *in, size_t inlen,
                  const u8 *key, size_t keylen)
{
    struct gost_streebog_state state;
    u8 x_key[STREEBOG_BLOCK_SIZE];
    u8 i_hash[STREEBOG256_DIGEST_SIZE];
    u8 pad_key[STREEBOG_BLOCK_SIZE];
    int i;

    /* 1. Нормализация ключа до 64 байт */
    if (keylen > STREEBOG_BLOCK_SIZE) {
        gost_streebog256(x_key, key, keylen);
        memset(x_key + STREEBOG256_DIGEST_SIZE, 0,
               STREEBOG_BLOCK_SIZE - STREEBOG256_DIGEST_SIZE);
    } else {
        memcpy(x_key, key, keylen);
        if (keylen < STREEBOG_BLOCK_SIZE)
            memset(x_key + keylen, 0, STREEBOG_BLOCK_SIZE - keylen);
    }

    /* 2. inner_hash: H((K^ipad) || msg) */
    for (i = 0; i < STREEBOG_BLOCK_SIZE; ++i)
        pad_key[i] = x_key[i] ^ 0x36;

    gost_streebog256_init(&state);
    gost_streebog256_update(&state, pad_key, STREEBOG_BLOCK_SIZE);
    gost_streebog256_update(&state, in, inlen);
    gost_streebog256_final(&state, i_hash);

    /* 3. outer_hash: H((K^opad) || inner_hash) */
    for (i = 0; i < STREEBOG_BLOCK_SIZE; ++i)
        pad_key[i] = x_key[i] ^ 0x5c;

    gost_streebog256_init(&state);
    gost_streebog256_update(&state, pad_key, STREEBOG_BLOCK_SIZE);
    gost_streebog256_update(&state, i_hash, STREEBOG256_DIGEST_SIZE);
    gost_streebog256_final(&state, out);

    /* Очистка ключевой информации */
    memzero_explicit(x_key, sizeof(x_key));
    memzero_explicit(pad_key, sizeof(pad_key));
    memzero_explicit(i_hash, sizeof(i_hash));
}
\end{minted}


\subsubsection{Блочный шифр <<Кузнечик>> и режим MGM}

В отличие от алгоритма хэширования реализация блочного шифра ГОСТ Р 34.12-2015 (<<Кузнечик>>) отсутствует в ядре Linux. Так что было принято решение разработать свой собственный загружаемый модуль ядра, который его реализует.\\ ~\SeeApp{app:modules}

Для обеспечения конфиденциальности и целостности данных, согласно требованиям протокола, используется режим аутентифицированного шифрования (AEAD). В соответствии с ГОСТ Р 34.13-2015~\cite{gost_34_12_2015} был выбран режим гаммирования с мультилинейным преобразованием — \texttt{MGM} (Multilinear Galois Mode).

\nlparagraph{Оптимизация производительности}

<<Кузнечик>> является шифром на основе \texttt{SP}-сети. Прямая программная реализация линейного преобразования $L$ (умножение вектора на матрицу в поле $GF(2^8)$) обладает низкой производительностью. Для обеспечения пропускной способности, сравнимой с оригинальной версией \texttt{WireGuard}, была применена техника предварительно вычисленных таблиц.\\

Преобразование $L \circ S$ над 128-битным блоком заменяется на сумму результатов выборки из таблиц:
$$
    L(S(a_0, \dots, a_{15})) = \bigoplus_{i=0}^{15} T_i[a_i]
$$
где $T_i$ — таблицы размером $256 \times 128$ бит.

При написании модуля были применены следующие техники оптимизации:
\begin{itemize}
    \item \textbf{Предварительный расчет таблиц:} использование 16 таблиц для шифрования и 16 для расшифрования;
    \item \textbf{Развертка циклов:} полная развертка циклов внутри раундов шифрования;
    \item \textbf{Невыравненный доступ к буферам:} использование макросов \texttt{get\_unaligned} / \texttt{put\_unaligned} для работы с сетевыми буферами без предварительного копирования.
\end{itemize}

\nlparagraph{Реализация режима MGM}

Режим \texttt{MGM} работает по схеме \texttt{Encrypt-then-MAC}. Ключевой операцией, влияющей на скорость, является умножение в поле Галуа $GF(2^{128})$ при вычислении имитовставки. В реализации задействована оптимизированная функция ядра \texttt{gf128mul\_x\_bbe} и 64-битные инструкции XOR.

\begin{minted}[tabsize=4]{c}
/* Макрос для оптимизированного преобразования LSX с использованием T-таблиц.
 * Выполняет XOR результатов выборки из 16 таблиц (по одной на каждый байт).
 */
static __always_inline void LSX(u8 *a, const u8 *b, const u8 *c)
{
    u64 r0, r1;
    const u64 *t;

    /* Инициализация первой итерации (байт 0) */
    t  = (const u64 *)&kuz_table[0][(b[0] ^ c[0]) * 16];
    r0 = t[0];
    r1 = t[1];

    /* Развернутый цикл XOR для оставшихся 15 байт */
#define LSX_XOR(i) \
    t = (const u64 *)&kuz_table[i][(b[i] ^ c[i]) * 16]; \
    r0 ^= t[0]; \
    r1 ^= t[1];

    LSX_XOR(1);  LSX_XOR(2);  LSX_XOR(3);  LSX_XOR(4);
    LSX_XOR(5);  LSX_XOR(6);  LSX_XOR(7);  LSX_XOR(8);
    LSX_XOR(9);  LSX_XOR(10); LSX_XOR(11); LSX_XOR(12);
    LSX_XOR(13); LSX_XOR(14); LSX_XOR(15);

    /* Сохранение результата в выходной буфер как u64 (оптимизация) */
    ((u64 *)a)[0] = r0;
    ((u64 *)a)[1] = r1;
#undef LSX_XOR
}

/* Функция шифрования одного блока (ECB).
 * Использует предварительно развернутые ключи и макрос LSX.
 */
static void kuznyechik_encrypt(struct crypto_tfm *tfm, u8 *out, const u8 *in)
{
    const struct crypto_kuznyechik_ctx *ctx = crypto_tfm_ctx(tfm);
    const u64 *keys = (const u64 *)ctx->key;
    u8 temp[KUZNYECHIK_BLOCK_SIZE] __aligned(sizeof(u64));
    u64 *o = (u64 *)out;
    u64 *t = (u64 *)temp;

    /* 9 раундов преобразований LSX */
    LSX(temp, (u8 *)(keys + 0), in);    /* Раунд 1 */
    LSX(temp, (u8 *)(keys + 2), temp);  /* Раунд 2 */
    LSX(temp, (u8 *)(keys + 4), temp);  /* Раунд 3 */
    /* ... */
    LSX(temp, (u8 *)(keys + 16), temp); /* Раунд 9 */

    /* Последний раунд: только сложение с ключом (X-преобразование) */
    o[0] = t[0] ^ keys[18];
    o[1] = t[1] ^ keys[19];
}

/* Быстрое обновление тега аутентификации MGM.
 * Использует встроенную функцию ядра для умножения в поле Галуа.
 */
static inline void mgm_mul_add_tag_fast(be128 *tag, const u8 *data,
                                        const be128 *h_i)
{
    be128 d_block;
    /* Оптимизация: чтение 64-битными словами, обработка невыровненных данных */
    const u64 *dp = (const u64 *)data;
    u64 *tp = (u64 *)&d_block;

    tp[0] = get_unaligned(&dp[0]);
    tp[1] = get_unaligned(&dp[1]);

    /* Аппаратно-оптимизированное умножение: d_block = d_block * h_i in GF(2^128) */
    gf128mul_x_bbe(&d_block, h_i);

    /* Аккумуляция результата в тег (XOR) */
    kuz_be128_xor(tag, tag, &d_block);
}
\end{minted}


\nlparagraph{Слой абстракции \texttt{AEAD}}

По аналогии с другими алгоритмами был также написан слой абстракции \texttt{AEAD} для шифрования и аутентификации данных. (\texttt{gost/gost\_kuznyechik.c})

\begin{minted}[tabsize=4]{c}
/* Вспомогательная функция для выполнения операций AEAD.
 * Инкапсулирует работу со структурами ядра scatterlist и aead_request.
 */
static int do_aead_op(struct crypto_aead *tfm, u8 *dst, const u8 *src, size_t len,
                      const u8 *ad, size_t ad_len, const u8 *nonce, bool encrypt)
{
    struct aead_request *req;
    struct scatterlist sg_src[2], sg_dst[2];
    int ret;

    /*
     * Так как WireGuard обрабатывает пакеты в контексте SoftIRQ.
     * Использование GFP_KERNEL здесь привело бы к "sleep inside atomic section".
     * Поэтому обязательно используем GFP_ATOMIC.
     */
    req = aead_request_alloc(tfm, GFP_ATOMIC);
    if (!req)
        return -ENOMEM;

    /*
     * Формирование цепочек scatterlist (SG).
     * API ядра требует, чтобы ассоциированные данные (AD) шли перед
     * шифруемыми данными в виртуальном представлении памяти.
     */
    if (ad_len > 0) {
        /* Формируем входной SG: [AD] -> [SRC] */
        sg_init_table(sg_src, 2);
        sg_set_buf(&sg_src[0], ad, ad_len);
        sg_set_buf(&sg_src[1], src, len);

        /* Формируем выходной SG: [AD] -> [DST] */
        sg_init_table(sg_dst, 2);
        sg_set_buf(&sg_dst[0], ad, ad_len);
        /* При шифровании выходной буфер должен вмещать данные + тег (16 байт) */
        sg_set_buf(&sg_dst[1], dst, len + (encrypt ? KUZNYECHIK_MGM_TAG_SIZE : 0));

        aead_request_set_crypt(req, sg_src, sg_dst, len, (u8 *)nonce);
        aead_request_set_ad(req, ad_len);
    } else {
        /* Упрощенный случай без AD (линейное отображение) */
        sg_init_one(sg_src, src, len);
        sg_init_one(sg_dst, dst, len + (encrypt ? KUZNYECHIK_MGM_TAG_SIZE : 0));

        aead_request_set_crypt(req, sg_src, sg_dst, len, (u8 *)nonce);
        aead_request_set_ad(req, 0);
    }

    /* Выполняем операцию. В данном случае вызов синхронный,
     * так как используется программная реализация шифра.
     */
    if (encrypt)
        ret = crypto_aead_encrypt(req);
    else
        ret = crypto_aead_decrypt(req);

    aead_request_free(req);
    return ret;
}

/* Публичный интерфейс для шифрования, используемый в send.c */
int gost_kuznyechik_mgm_encrypt(struct gost_kuznyechik_mgm_ctx *ctx,
                                u8 *dst, const u8 *src, size_t src_len,
                                const u8 *ad, size_t ad_len, const u8 *nonce)
{
    if (unlikely(!ctx->tfm))
        return -EINVAL;
    return do_aead_op(ctx->tfm, dst, src, src_len, ad, ad_len, nonce, true);
}
\end{minted}


\subsubsection{Протокол выработки общего ключа VKO ГОСТ Р 34.10-2012}

В качестве замены алгоритму \texttt{ECDH} (\texttt{Curve25519}) реализован протокол \texttt{VKO} (Выработка Ключа Обмена) на базе эллиптических кривых. Модуль \texttt{ec256\_vko} регистрируется в ядре как примитив типа \texttt{KPP} (Key-agreement Protocol Primitive). ~\SeeApp{app:modules}

\nlparagraph{Параметры алгоритма}

Используется 256-битная эллиптическая кривая с параметрами \texttt{id-tc26-gost-3410-2012-256-paramSetA}. Вычисление общей точки производится по формуле:

$$
    K = (h \cdot d_{local}) \times Q_{remote}
$$

Для арифметических операций с большими числами (\texttt{BigInt}) адаптирована библиотека \texttt{crypto/ecc.c}, обеспечивающая выполнение операций за постоянное время, что защищает от атак по побочным каналам.

\nlparagraph{Программная реализация}

Особенностью программной реализации является работа с 64-байтными открытыми ключами (две координаты по 32 байта) и поддержка параметра \texttt{UKM} (\texttt{User Keying Material}). Функция \texttt{gost\_vko\_compute\_value} выполняет декодирование ключей, скалярное умножение и хэширование результата для получения сессионного ключа.

\begin{minted}[tabsize=4]{c}
/* Основная функция вычисления общего секрета (VKO).
 * Выполняет роль callback-функции для KPP API ядра.
 */
static int gost_vko_compute_value(struct kpp_request *req)
{
    struct crypto_kpp *tfm = crypto_kpp_reqtfm(req);
    struct gost_vko_ctx *ctx = gost_vko_get_ctx(tfm);
    u64 scalar[GOST_EC256_NDIGITS];

    /* ... */

    if (req->src) {
        /* --- Режим вычисления общего секрета (Shared Secret) --- */

        /* 1. Декодирование публичного ключа партнера (точка Q).
         * Ключ приходит в формате Big-Endian (64 байта), преобразуем
         * во внутренний формат VLI (массив u64).
         */
        /* Координата X */
        gost_vli_from_be(public_key_buf, req->src, GOST_EC256_NDIGITS);
        /* Координата Y (смещение 32 байта) */
        gost_vli_from_be(public_key_buf + GOST_EC256_NDIGITS,
                         req->src + GOST_EC256_KEY_SIZE, GOST_EC256_NDIGITS);

        /* 2. Вычисление эффективного скаляра S.
         * Согласно ГОСТ Р 34.10-2012:
         * Если передан UKM (User Keying Material), то S = (d * UKM) mod n.
         * Иначе S = d (собственный закрытый ключ).
         */
        if (ctx->params.ukm_size > 0) {
            gost_vli_mod_mult(scalar, ctx->params.key, ctx->params.ukm,
                              ctx->curve->n, GOST_EC256_NDIGITS);
        } else {
            memcpy(scalar, ctx->params.key, sizeof(scalar));
        }

        /* 3. Скалярное умножение точки кривой.
         * Вычисляем точку K = S * Q_peer.
         * Функция использует алгоритм с защитой от атак по времени.
         */
        gost_ec256_point_mul(shared_secret_x,     /* Результат (Координата X) */
                             temp_y,              /* Результат Y (отбрасывается) */
                             public_key_buf,      /* Peer X */
                             public_key_buf + GOST_EC256_NDIGITS, /* Peer Y */
                             scalar, ctx->curve);

        /* 4. Формирование KEK (Key Encryption Key).
         * Согласно алгоритму VKO, общим секретом является хэш от
         * координаты X полученной точки: KEK = Hash(K.x)
         */
        /* ... Вызов хэширования streebog256(shared_secret_x) ... */
    }
    else {
        /* --- Режим генерации собственного публичного ключа --- */
        /* Q = d * G, где G - базовая точка кривой */
        gost_ec256_point_mul(public_key_buf,
                             public_key_buf + GOST_EC256_NDIGITS,
                             ctx->curve->g.x, ctx->curve->g.y,
                             ctx->params.key, ctx->curve);
    }

    /* ... Копирование результата в req->dst ... */
    return 0;
}
\end{minted}


\subsection{Интеграция в сетевой протокол WireGuard}

Дальнейшим шагом разработки \texttt{WireGost} стало внедрение ранее реализованных криптографических примитивов в сетевой протокол \texttt{WireGuard}. Этот шаг оказался довольно сложным, поскольку пришлось переписывать значительную часть кода, связанного с обработкой пакетов, а также общением с пользовательским пространством через \texttt{Netlink}.

\subsubsection{Адаптация структур данных}

Первым этапом портирования стало изменение размера криптографических ключей, так как реализованный алгоритм \texttt{VKO} использует ключи размером 64 байта, что в 2 раза больше, чем в оригинальном протоколе:
\begin{itemize}
    \item \textbf{Публичные ключи:} увеличены с 32 до 64 байт (нескомпрессированная точка);
    \item \textbf{Размер пакета:} изменения отражены в структурах \texttt{message\_handshake\_initiation} и \texttt{message\_handshake\_response} в файле \texttt{messages.h}.
\end{itemize}

Увеличение заголовков пакетов рукопожатия было учтено при расчете \texttt{MTU} туннеля, чтобы избежать фрагментации пакетов на этапе установления соединения.

\subsubsection{Модификация машины состояний Noise}

Ядро протокола (\texttt{noise.c}) было адаптировано для использования API разработанных модулей. Функция \texttt{mix\_dh}, отвечающая за обновление цепочки ключей (chaining key), переписана для вызова \texttt{gost\_ec256\_dh}.
Вместо прямой математической операции теперь происходит обращение к KPP API ядра:

\begin{minted}[tabsize=4]{c}
/* Адаптированная функция mix_dh для VKO ГОСТ */
static bool mix_dh(u8 *chaining_key, u8 *key, struct wg_peer *peer)
{
    u8 shared_secret[NOISE_HASH_LEN];
    /* Вызов KPP API для вычисления VKO */
    if (!gost_ec256_dh(shared_secret,
                      peer->handshake.static_private,
                      peer->handshake.remote_static,
                      peer->handshake.ukm,
                      peer->handshake.ukm_len))
        return false;

    /* KDF на базе HMAC-Streebog */
    kdf(chaining_key, key, NULL, shared_secret, ...);
    memzero_explicit(shared_secret, NOISE_HASH_LEN);
    return true;
}
\end{minted}

\subsubsection{Взаимодействие через Netlink}

Так как был изменен размер ключа, в протоколе \texttt{Netlink} были обновлены атрибуты (\texttt{WGDEVICE\_A\_PEER\_PUBLIC\_KEY}, \texttt{WGDEVICE\_A\_PRIVATE\_KEY}), а в драйвер добавлены проверки, не позволяющие конфигурировать интерфейс ключами некорректной длины.

\subsubsection{Инициализация и управление зависимостями}

Из-за того, что в модуль \texttt{WireGost} был добавлен слой абстракции, для каждого из используемых криптографических алгоритмов необходимо было выполнить инициализацию и деинициализацию.
Для этого в \texttt{main.c} были добавлены вызовы \texttt{gost\_streebog\_init\_module}, \texttt{gost\_kuznyechik\_init\_tfms}, \texttt{gost\_ec256\_init\_module}.\\

Более того, был расширен набор тестов, которые выполняются при сборке в режиме \texttt{DEBUG}. Были добавлены встроенные тесты, проверяющие корректность работы алгоритмов на эталонных векторах:
\begin{minted}[tabsize=4]{c}
#ifdef DEBUG
	RUN_TEST("Streebog-256", self_test_streebog);
	RUN_TEST("HMAC-Streebog-256", self_test_hmac_streebog);
	RUN_TEST("Kuznyechik-MGM", self_test_kuznyechik_mgm);
	RUN_TEST("GOST R 34.10-2012 (EC512)", self_test_gost_vko_generated);

	pr_info("WireGost crypto self-tests completed successfully.\n");
#endif
\end{minted}

Сами тесты были вынесены в \texttt{selftest/selftest.c} к остальным файлам самотестирования. \\

Для упрощения добавления новых тестов был создан макрос \texttt{RUN\_TEST}, который автоматически генерирует код для запуска теста.
\begin{minted}[tabsize=4]{c}
#define RUN_TEST(name, func)            \
do                                      \
{                                       \
	pr_info("Testing " name "... ");    \
	if (func())                         \
		pr_cont("PASSED\n");            \
	else                                \
	{                                   \
		pr_cont("FAILED\n");            \
		ret = -EFAULT;                  \
		goto err_tests;                 \
	}                                   \
} while (0)
\end{minted}

Сами тесты представляют собой набор эталонных значений, определенных в ГОСТ соответствующего протокола.~\cite{gost_34_11_2012} Ниже представлен пример теста для алгоритма \texttt{Streebog-256}.
\begin{minted}[tabsize=4]{c}
static bool self_test_streebog(void)
{
	u8          digest[32];
	const char *msg          = "01234567890123456789012345678901"
		                       "2345678901234567890123456789012";
	const u8    expected[32] = {0x9d, 0x15, 0x1e, 0xef, 0xd8, 0x59,
	                            0x0b, 0x89, 0xda, 0xa6, 0xba, 0x6c,
	                            0xb7, 0x4a, 0xf9, 0x27, 0x5d, 0xd0,
								0x51, 0x02, 0x6b, 0xb1, 0x49, 0xa4,
								0x52, 0xfd, 0x84, 0xe5, 0xe5, 0x7b,
								0x55, 0x00};

	gost_streebog256(digest, (u8 *)msg, 63);

	if (memcmp(digest, expected, 32) != 0)
	{
		pr_err("Streebog-256 digest mismatch\n");
		return false;
	}
	return true;
}
\end{minted}

\subsubsection{Конфигурация протокола Noise\_GOST}

В файле \texttt{noise.c} определена уникальная строка идентификации протокола:
\begin{minted}{c}
"Noise_IKpsk2_GOST_EC512_KuznyechikMGM_Streebog256"
\end{minted}
Это гарантирует криптографическую изоляцию сессий \texttt{WireGost} от оригинального \texttt{WireGuard}, предотвращая попытки соединения с несовместимыми клиентами. Функции деривации ключей (\texttt{KDF}) переведены на использование конструкции HMAC-Streebog256.\\

\nlparagraph{Функция формирования ключа (KDF)}

Стандартная реализация \texttt{KDF} в протоколе Noise базируется на \texttt{HKDF} (HMAC-based Key Derivation Function). В разработанном модуле функция была адаптирована для использования \texttt{HMAC} на базе ГОСТ Р 34.11-2012:

\begin{minted}[tabsize=4]{c}
static void kdf(u8 *first_dst, u8 *second_dst, u8 *third_dst,
                const u8 *data, size_t first_len, size_t second_len,
                size_t third_len, size_t data_len,
                const u8 chaining_key[NOISE_HASH_LEN])
{
    u8 output[NOISE_HASH_LEN + 1];
    u8 secret[NOISE_HASH_LEN];

    /* Шаг 1: Извлечение */
    gost_hmac256(secret, data, data_len, chaining_key, NOISE_HASH_LEN);

    /* Шаг 2: Расширение для первого ключа */
    if (first_dst && first_len) {
        output[0] = 1;
        gost_hmac256(output, output, 1, secret, NOISE_HASH_LEN);
        memcpy(first_dst, output, first_len);
    }

    /* Расширение для второго ключа */
    if (second_dst && second_len) {
        output[NOISE_HASH_LEN] = 2;
        gost_hmac256(output, output, NOISE_HASH_LEN + 1, secret, NOISE_HASH_LEN);
        memcpy(second_dst, output, second_len);
    }

    /* ... (аналогично для третьего ключа) ... */

    /* Очистка ключевого материала из памяти */
    memzero_explicit(secret, NOISE_HASH_LEN);
    memzero_explicit(output, NOISE_HASH_LEN + 1);
}
\end{minted}

\nlparagraph{Обертка шифрования сообщений}

Для интеграции блочного шифра <<Кузнечик>> в режиме \texttt{MGM} в поток обработки пакетов были реализованы функции-обертки \texttt{message\_encrypt} и \texttt{message\_decrypt}. Они выполняют следующие задачи:

\begin{enumerate}
    \item инициализация контекста шифрования сессионным ключом;
    \item вызов процедуры \texttt{AEAD}-шифрования;
    \item обновление текущего хэша рукопожатия шифртекстом, что обеспечивает криптографическое связывание всех сообщений сессии.
\end{enumerate}

\begin{minted}[tabsize=4]{c}
static void message_encrypt(u8 *dst_ciphertext, const u8 src_plaintext,
                            size_t src_len, u8 key[NOISE_SYMMETRIC_KEY_LEN],
                            u8 hash[NOISE_HASH_LEN])
{
    u8 nonce[16] = {0};
    struct gost_kuznyechik_mgm_ctx ctx;
    memset(&ctx, 0, sizeof(ctx));

    /* Установка ключа и шифрование */
    if (likely(gost_kuznyechik_mgm_set_key(&ctx, key) == 0))
    {
        gost_kuznyechik_mgm_encrypt(&ctx, dst_ciphertext, src_plaintext,
                                    src_len, hash, NOISE_HASH_LEN, nonce);
        gost_kuznyechik_mgm_free_ctx(&ctx);
    }

    /* MixHash: H = Hash(H || Ciphertext) */
    mix_hash(hash, dst_ciphertext, noise_encrypted_len(src_len));
}

static bool message_decrypt(u8 *dst_plaintext, const u8 *src_ciphertext,
                            size_t src_len, u8 key[NOISE_SYMMETRIC_KEY_LEN],
                            u8 hash[NOISE_HASH_LEN])
{
	u8							   nonce[16] = {0};
	bool						   ret		 = false;
	struct gost_kuznyechik_mgm_ctx ctx;
	memset(&ctx, 0, sizeof(ctx));
	if (likely(gost_kuznyechik_mgm_set_key(&ctx, key) == 0))
	{
		if (gost_kuznyechik_mgm_decrypt(&ctx, dst_plaintext, src_ciphertext,
		                                src_len, hash, NOISE_HASH_LEN, nonce)
		    == 0)
		{
			ret = true;
		}
		gost_kuznyechik_mgm_free_ctx(&ctx);
	}
	if (!ret)
		return false;
	mix_hash(hash, src_ciphertext, src_len);
	return true;
}
\end{minted}

\nlparagraph{Инкапсуляция и шифрование}

Процесс отправки пакета реализован в файле \texttt{send.c}.\\

Перед шифрованием данные должны быть подготовлены:
\begin{enumerate}
    \item \textbf{Выравнивание:} для скрытия точной длины передаваемых данных и соблюдения требований блочного шифра, к пакету добавляется случайное заполнение. Размер выравнивания рассчитывается так, чтобы итоговая длина блока соответствовала требованиям \texttt{MTU} и границам блока шифра;
    \item \textbf{Формирование заголовка:} создается структура \texttt{message\_data}, содержащая тип сообщения (\texttt{MESSAGE\_DATA}), индекс ключа получателя и 64-битный счетчик (Nonce), используемый для защиты от атак повтора.
\end{enumerate}

Функция \texttt{encrypt\_packet} выполняет эти действия и вызывает криптографический примитив:

\begin{minted}[tabsize=4]{c}
static bool encrypt_packet(struct sk_buff *skb, struct noise_keypair *keypair)
{
    unsigned int padding_len, plaintext_len, trailer_len;
    struct message_data header;
    /* ... */
    u8 nonce[16] = {0};
    /* Расчет выравнивания и общей длины текста */
    padding_len   = calculate_skb_padding(skb);
    /* Длина трейлера включает имитовставку (Tag) */
    trailer_len   = padding_len + noise_encrypted_len(0);
    plaintext_len = skb->len + padding_len;

    /* ... подготовка буфера skb (skb_cow_data) ... */

    /* Формирование заголовка */
    header              = (struct message_data *)skb_push(skb, sizeof(*header));
    header->header.type = cpu_to_le32(MESSAGE_DATA);
    header->key_idx     = keypair->remote_index;
    header->counter     = cpu_to_le64(PACKET_CB(skb)->nonce);

    /* Использование 64-битного счетчика как части Nonce для MGM */
    memcpy(nonce, &header->counter, 8);

    /* ... линеаризация буфера ... */

    /* Шифрование "на месте" (in-place) с использованием Кузнечик-MGM */
    if (gost_kuznyechik_mgm_encrypt(&keypair->sending.mgm_ctx,
                                    skb->data + sizeof(struct message_data),
                                    skb->data + sizeof(struct message_data),
                                    plaintext_len, NULL, 0, nonce))
    {
        return false;
    }

    return true;
}
\end{minted}

\nlparagraph{Инкапсуляция и шифрование}

При получении пакета (файл \texttt{receive.c}) драйвер сначала определяет тип сообщения. Если это \texttt{MESSAGE\_DATA}, управление передается функции \texttt{decrypt\_packet}.\\

Процедура расшифрования включает:
\begin{enumerate}
    \item \textbf{Проверку счетчика:} извлечение значения \texttt{counter} из заголовка и проверка, не является ли пакет дубликатом (\texttt{Replay Attack Protection});
    \item \textbf{Расшифрование с аутентификацией:} вызов функции \texttt{gost\_kuznyechik\_mgm\_decrypt}. Режим \texttt{MGM} работает по схеме \texttt{AEAD}, поэтому проверка целостности (сверка имитовставки) происходит одновременно с расшифрованием. Если имитовставка не совпадает, пакет немедленно отбрасывается.
\end{enumerate}

\begin{minted}[tabsize=4]{c}
static bool decrypt_packet(struct sk_buff *skb, struct noise_keypair keypair)
{
    u8 nonce[16] = {0};
    /* ... */

    /* Валидация состояния ключа и счетчиков */
    if (unlikely(!READ_ONCE(keypair->receiving.is_valid) || ... )) {
        return false;
    }

    /* Извлечение Nonce из заголовка пакета */
    PACKET_CB(skb)->nonce = le64_to_cpu(((struct message_data *)skb->data)->counter);
    memcpy(nonce, &((struct message_data *)skb->data)->counter, 8);

    /* ... подготовка буфера ... */

    /* Дешифрование и проверка имитовставки */
    if (gost_kuznyechik_mgm_decrypt(&keypair->receiving.mgm_ctx,
                                    skb->data + sizeof(struct message_data),
                                    skb->data + sizeof(struct message_data),
                                    skb->len - sizeof(struct message_data),
                                    NULL, 0, nonce))
    {
        return false;
    }

    /* Удаление заголовка WireGuard и имитовставки (trim) */
    if (pskb_trim(skb, skb->len - NOISE_AUTHTAG_LEN))
        return false;
    skb_pull(skb, sizeof(struct message_data));

    return true;
}
\end{minted}

\subsection{Разработка инструментария конфигурации и тестирования}

Поскольку внедрение алгоритмов описанных в ГОСТ (в частности ГОСТ 34.10-2012 и ГОСТ 34.11-2012)~\cite{gost_34_10_2012, gost_34_11_2012} потребовало изменения длины криптографических ключей (увеличение с 32 до 64 байт для закрытого ключа и до 128 байт для открытого), стандартные утилиты пространства пользователя, такие как \texttt{wg} и \texttt{wg-quick}, оказались несовместимы с разработанным модулем \texttt{WireGost}.\\

Для отладки и тестирования взаимодействия с модулем ядра была разработана специализированная утилита командной строки \texttt{wg\_gost\_cli}. \\

\subsubsection{Архитектура утилиты конфигурации}

Утилита реализована на языке C и взаимодействует с ядром через интерфейс \texttt{Generic Netlink}. В отличие от использования библиотеки \texttt{libmnl}, которая применяется в оригинальном \texttt{WireGuard}, здесь реализовано прямое формирование сообщений \texttt{Netlink} для обеспечения полного контроля над структурой атрибутов. \\

\begin{minted}[tabsize=4]{c}
/* testing/wg_gost_cli.c */
#define WG_GENL_NAME "wiregost"

/* Размер закрытого ключа VKO (512 бит) /
#define GOST_PRIV_KEY_LEN 64
/ Размер открытого ключа (точка на эллиптической кривой, 64+64 байт) */
#define GOST_PUB_KEY_LEN 128

/* ... определения атрибутов Netlink (WGDEVICE_A_, WGPEER_A_) ... */
\end{minted}

\subsubsection{Инициализация интерфейса и генерация ключей}

Основной задачей утилиты является инициализация сетевого интерфейса и генерация ключевой пары. Утилита поддерживает команду \texttt{init}, которая выполняет следующие действия:

\begin{enumerate}
    \item преобразование имени интерфейса в индекс (\texttt{if\_nametoindex});
    \item генерация случайного закрытого ключа;
    \item отправка команды \texttt{WG\_CMD\_SET\_DEVICE} с атрибутом \texttt{WGDEVICE\_A\_PRIVATE\_KEY};
    \item отправка запроса \texttt{WG\_CMD\_GET\_DEVICE} для получения вычисленного ядром открытого ключа.
\end{enumerate}

Реализация отправки конфигурации:

\begin{minted}[tabsize=4]{c}
int cmd_init(int sock, int family_id, int argc, char *argv)
{
    /* ... инициализация переменных ... */

    /* Генерация тестового ключа (обнуление старшего байта для валидности) */
    gen_random_key(priv, GOST_PRIV_KEY_LEN, getpid() + port);

    /* Формирование запроса SET_DEVICE */
    req.n.nlmsg_len   = NLMSG_LENGTH(GENL_HDRLEN);
    req.n.nlmsg_type  = family_id;
    req.n.nlmsg_flags = NLM_F_REQUEST | NLM_F_ACK;
    req.g.cmd         = WG_CMD_SET_DEVICE;

    /* Добавление атрибутов: индекс, приватный ключ, порт */
    add_attr(&req, WGDEVICE_A_IFINDEX, &ifindex, sizeof(ifindex));
    add_attr(&req, WGDEVICE_A_PRIVATE_KEY, priv, GOST_PRIV_KEY_LEN);
    add_attr(&req, WGDEVICE_A_LISTEN_PORT, &port, sizeof(port));

    /* Отправка в ядро */
    if (send(sock, &req, req.n.nlmsg_len, 0) < 0) { /* ... */ }

    /* ... чтение ответа и запрос публичного ключа ... */
}
\end{minted}

\subsubsection{Настройка пиров}

Вторая ключевая функция --- добавление доверенных узлов (пиров). Команда \texttt{peer} формирует сложное вложенное сообщение \texttt{Netlink}, содержащее структуру \texttt{WGDEVICE\_A\_PEERS}, внутри которой находится список пиров с их параметрами: открытый ключ (128 байт), конечная точка (Endpoint) и разрешенные IP-адреса.\\

Особенностью реализации является корректное формирование вложенных атрибутов (\texttt{NLA\_F\_NESTED}), что необходимо для корректного парсинга со стороны ядра.

\begin{minted}[tabsize=4]{c}
int cmd_peer(int sock, int family_id, int argc, char *argv)
{
    /* ... парсинг аргументов командной строки ... */
    /* Начало вложенного атрибута списка пиров */
    struct nlattr *peers = (struct nlattr *)((char *)&req + req.n.nlmsg_len);
    peers->nla_type      = WGDEVICE_A_PEERS | NLA_F_NESTED;
    req.n.nlmsg_len     += NLA_HDRLEN;

    /* Начало атрибута конкретного пира */
    struct nlattr *peer0 = (struct nlattr *)((char *)&req + req.n.nlmsg_len);
    peer0->nla_type      = 0 | NLA_F_NESTED; /* Индекс 0 */
    req.n.nlmsg_len     += NLA_HDRLEN;

    /* Установка параметров пира */
    add_attr(&req, WGPEER_A_PUBLIC_KEY, pub, GOST_PUB_KEY_LEN);
    add_attr(&req, WGPEER_A_ENDPOINT, &ep4, sizeof(ep4));

    /* ... добавление разрешенных IP (AllowedIPs) ... */

    /* Финализация длин вложенных атрибутов */
    peer0->nla_len = (char *)&req + req.n.nlmsg_len - (char *)peer0;
    peers->nla_len = (char *)&req + req.n.nlmsg_len - (char *)peers;

    /* Отправка конфигурации в ядро */
    if (send(sock, &req, req.n.nlmsg_len, 0) < 0) { /* ... */ }

    return 0;
}
\end{minted}

\subsection{Методология тестирования}

Для объективной оценки производительности разработанного решения \texttt{WireGost} и сравнения его с оригинальным протоколом \texttt{WireGuard}, с \texttt{go} версией \texttt{ru-WireGuard} на ГОСТ протоколах, а также с эталонной пропускной способностью канала, был разработан автоматизированный комплекс скриптов на языке \texttt{Bash}.~\SeeApp{app:test_benchmark}

\subsubsection{Тестовая среда и топология}

Тестирование проводилось в изолированной среде с использованием механизма сетевых пространств имен Linux (\texttt{Network Namespaces}). Это позволяет эмулировать полноценное сетевое взаимодействие между двумя узлами на одной физической машине, исключая влияние внешнего сетевого оборудования, но сохраняя полный путь прохождения пакетов через сетевой стек ядра.

Топология тестового стенда формируется функцией \texttt{setup\_namespaces} в библиотеке \texttt{lib\_benchmark.sh}:
\begin{itemize}
    \item создаются два пространства имен: \texttt{ns1} (клиент) и \texttt{ns2} (сервер);
    \item они соединяются парой виртуальных ethernet-интерфейсов (\texttt{veth}), эмулирующих физический канал связи (Underlay network, подсеть 10.0.0.0/24);
    \item поверх физического канала поднимаются VPN-интерфейсы (Overlay network, подсеть 192.168.1.0/24).
\end{itemize}

\begin{minted}[tabsize=4]{bash}
setup_namespaces() {
    echo -e "${GREEN}[Setup] Creating Namespaces and VETH...${NC}"
    ip netns add ns1
    ip netns add ns2

    ip link add veth1 type veth peer name veth2
    ip link set veth1 netns ns1
    ip link set veth2 netns ns2

    # Настройка транспортной сети (Underlay)
    ip netns exec ns1 bash -c "ip addr add 10.0.0.1/24 dev veth1; \
        ip link set veth1 up"
    ip netns exec ns2 bash -c "ip addr add 10.0.0.2/24 dev veth2; \
        ip link set veth2 up"
}
\end{minted}

\subsubsection{Архитектура тестового комплекса}

Комплекс построен по модульному принципу. Основная логика измерений вынесена в библиотеку \texttt{lib\_benchmark.sh}, которая предоставляет унифицированный интерфейс для запуска тестов. Специфичные для реализации настройки (настройка WireGuard, ruWireGuard, WireGost или <<чистого>> канала) вынесены в отдельные сценарии, переопределяющие функцию \texttt{setup\_vpn\_impl}.\\

Сценарий тестирования включает следующие этапы:
\begin{enumerate}
    \item \textbf{Тест пропускной способности TCP и эффективности CPU:} \\
    Для тестирования используется утилита \texttt{iperf3}. Кроме измерения скорости (Mbit/s), рассчитывается <<стоимость>> передачи данных для процессора.\\
    Метрика эффективности рассчитывается по формуле:

    $$
        Efficiency = \frac{Throughput_{Mbps}}{CPU_{usage}}
    $$
    Где $CPU_{usage}$ — суммарное потребление ресурсов ядра и пользователя за время теста.
    \begin{minted}[tabsize=4]{bash}
run_tcp_efficiency_test() {
    # ... запуск сервера и клиента iperf3 ...
    local cpu_diff=$((cpu_end - cpu_start))

    # Расчет удельной стоимости трафика
    CPU_COST=$(echo "scale=2; ($cpu_diff / $time_diff_sec)" | bc)
    EFFICIENCY=$(echo "scale=2; $TCP_MBPS / $CPU_COST" | bc)

    echo -e "    Throughput: ${CYAN}${TCP_MBPS} Mbps${NC}"
    echo -e "    Efficiency: ${CYAN}${EFFICIENCY}${NC}"
}
    \end{minted}

    \item \textbf{Тест максимальной пропускной способности UDP:} \\
    В отличие от TCP, протокол UDP не имеет встроенных механизмов контроля перегрузки. Запуск \texttt{iperf3} в режиме UDP с флагом \texttt{-b 0} (максимальная полоса) позволяет оценить <<чистую>> производительность канала и процент потери пакетов (\texttt{Packet Loss}), возникающих из-за переполнения буферов или нехватки процессорного времени на шифрование.

    \begin{minted}[tabsize=4]{bash}
run_udp_saturation_test() {
    echo -e "${GREEN}[Test] UDP Max Throughput...${NC}"
    ip netns exec ns1 iperf3 -c 192.168.1.2 -u -b 0 -t $TEST_DURATION \
        -J > "$LOG_DIR/iperf_udp.json"

    # ... парсинг JSON для получения пропускной способности и потерь ...

    echo -e "    Throughput: ${CYAN}${UDP_MBPS} Mbps${NC}"
    echo -e "    Packet Loss:${CYAN}${UDP_LOSS}%${NC}"
}
    \end{minted}

    \item \textbf{Стресс-тестирование PPS (Packets Per Second):} \\
    Отправка потока мелких UDP-пакетов (64 байта) для оценки накладных расходов на обработку заголовков, прерываний и вызовов криптографических функций на каждый пакет, что является узким местом для тяжеловесных алгоритмов шифрования.

    \begin{minted}[tabsize=4]{bash}
run_udp_pps_test() {
    echo -e "${GREEN}[Test] UDP PPS (64 byte packets)...${NC}"
    # Отправка пакетов по 64 байта
    ip netns exec ns1 iperf3 -c 192.168.1.2 -u -l 64 -b 2G -t 5 \
        -J > "$LOG_DIR/iperf_pps.json"

    # ... расчет среднего PPS ...
    echo -e "    Result:     ${CYAN}${PPS_RATE} pps${NC}"
}
    \end{minted}

    \item \textbf{Тест целостности данных:} \\
    Для проверки корректности реализации криптографических алгоритмов (особенно режимов сцепления блоков и обработки счетчиков) был разработан тест на целостность передаваемых данных. Для этого из \texttt{/dev/urandom} генерируется файл размером 512 Мб, происходит передача файла через туннель, после чего сверяются контрольные суммы \texttt{MD5} на стороне отправителя и получателя.

    \begin{minted}[tabsize=4]{bash}
run_file_integrity_test() {
    echo -e "${GREEN}[Test] File Integrity (${FILE_SIZE_MB} MB)...${NC}"
    # Генерация случайного файла
    dd if=/dev/urandom of="$test_file" bs=1M count=$FILE_SIZE_MB \
        status=none
    local orig_sum=$(md5sum "$test_file" | awk '{print $1}')

    # Передача через сокет (socat/nc) внутри туннеля
    # ... передача ...

    # Сверка хеш-сумм
    if [ "$orig_sum" == "$recv_sum" ]; then
        echo -e "    Result:     ${CYAN}PASS${NC}"
    else
        echo -e "    Result:     ${RED}FAIL${NC} (Checksum Mismatch)"
    fi
}
    \end{minted}

    \item \textbf{Тест задержки под нагрузкой (Latency):} \\
    Измерение времени отклика (\texttt{RTT}) с помощью утилиты \texttt{ping}, выполняемое параллельно с фоновой генерацией трафика (UDP-поток). Это позволяет оценить влияние очередей и времени шифрования на интерактивность соединения (эффект Bufferbloat).

    \begin{minted}[tabsize=4]{bash}
run_latency_test() {
    echo -e "${GREEN}[Test] Latency under Load...${NC}"
    # Фоновая нагрузка
    ip netns exec ns1 iperf3 -c 192.168.1.2 -u -b 100M -t 10 \
        >/dev/null 2>&1 &
    local load_pid=$!
    sleep 2
    # Измерение задержки
    ip netns exec ns1 ping -c 5 192.168.1.2 > "$LOG_DIR/latency.log"
    wait $load_pid || true
    # ... парсинг среднего RTT ...
    echo -e "    Result:     ${CYAN}${LATENCY_AVG} ms${NC}"
}
    \end{minted}
\end{enumerate}

\subsection{Результаты тестирования}

Тестирование производительности проводилось на аппаратной платформе со следующими характеристиками:
\begin{itemize}
    \item \textbf{Процессор:} Intel Core i7-1360P (13-е поколение, архитектура Raptor Lake-P), 16 потоков, максимальная тактовая частота 5.0 ГГц;
    \item \textbf{Оперативная память:} 16 ГБ LPDDR5;
    \item \textbf{ОС и Ядро:} Arch Linux, ядро Linux 6.18.1-zen.
\end{itemize}

Для минимизации влияния планировщика процессов и фоновых задач каждый тест запускался по 10 раз, в качестве результатов пошли усредненные показатели.

\subsubsection{Эталонные показатели (Baseline)}

Измерения <<чистого>> виртуального канала (\texttt{veth}) без шифрования показали теоретический максимум производительности сетевого стека в данном окружении.
\begin{itemize}
    \item \textbf{TCP пропускная способность:} $\approx 80.8$ Гбит/с;
    \item \textbf{UDP пропускная способность:} $\approx 9.6$ Гбит/с;
    \item \textbf{Стоимость CPU:} 237.96 \texttt{Jiffies}.
\end{itemize}
Такие высокие показатели обусловлены тем, что для передачи данных для вычисления эталонных показателей не требуется никаких криптографических преобразований.

\subsubsection{Сравнительный анализ реализаций WireGuard}

В таблице \ref{tab:benchmark_results} приведены сводные результаты тестирования оригинального протокола \texttt{WireGuard}, \texttt{ruWireGuard} и разработанной автором реализации \texttt{WireGost}.~\SeeApp{app:test_results}

\begin{table}[H]
    \centering
    \small
    \setlength{\tabcolsep}{3pt}
    \caption{Сводная таблица результатов тестирования производительности}
    \label{tab:benchmark_results}
    \begin{tabular}{|m{2.5cm}|c|c|c|c|c|}
        \hline
        \textbf{Метрика}               & \textbf{WireGuard} & \textbf{ruWireGuard} & \textbf{WireGost} & \textbf{$\Delta$ (WG)} & \textbf{$\Delta$ (ruWG)} \\ \hline
        TCP T-put \newline (Mbps)      & 5503.61            & 61.54                & 743.12            & $\downarrow 7.4\times$ & $\uparrow 12.1\times$    \\ \hline
        UDP T-put \newline (Mbps)      & 3553.11            & 39.97                & 845.72            & $\downarrow 4.2\times$ & $\uparrow 21.2\times$    \\ \hline
        UDP Loss \newline (\%)         & 0.1714\%           & 98.5195\%            & 13.1979\%         & $\downarrow 77\times$  & $\uparrow 7.5\times$     \\ \hline
        CPU Cost \newline (units)      & 499.05             & 1414.62              & 664.09            & $\downarrow 1.3\times$ & $\uparrow 2.1\times$     \\ \hline
        Efficiency \newline (Mbps/CPU) & 11.03              & 0.04                 & 1.11              & $\downarrow 9.9\times$ & $\uparrow 27.8\times$    \\ \hline
        Latency \newline (ms)          & 0.098              & 441.522              & 0.083             & $\uparrow 1.2\times$   & $\uparrow 5300\times$    \\ \hline
        \multicolumn{6}{l}{
                \begin{tabular}[c]{@{}l@{}}
                    \scriptsize \textit{*$\Delta$ (WG) — отношение WireGost к WireGuard;} \\
                    \scriptsize \textit{**$\Delta$ (ruWG) — отношение WireGost к ruWireGuard.} \\
                    \scriptsize Важно отметить, что при тестировании \texttt{ruWireGuard} было проведено не 10, а 11 тестов, так как один из них привел к аварийному \\
                    \scriptsize завершению. К сожалению, так как тестирование проводилось в автоматическом режиме, получить подробную информацию о причине \\
                    \scriptsize аварийного завершения не удалось, как и повторить условия, при которых оно возникло.
                \end{tabular}
            }
    \end{tabular}
\end{table}

На основе полученных данных можно сделать следующие выводы о характеристиках разработанного автором программного комплекса:

\nlparagraph{Сравнение \texttt{WireGost} с оригинальным \texttt{WireGuard}}

Как и ожидалось, внедрение российских криптографических алгоритмов привело к снижению пропускной способности. \texttt{WireGost} показал результат в 743 Мбит/с (TCP), что в 7.4 раза медленнее оригинального протокола (5.5 Гбит/с). Это объясняется фундаментальными различиями используемых примитивов:

\begin{itemize}
    \item \textbf{Природа шифра:} \texttt{WireGuard} использует потоковый шифр \texttt{ChaCha20}, который выполняет минимальное количество операций над памятью (\texttt{XOR} с гаммой) и идеально оптимизирован для программной реализации. \texttt{WireGost} использует блочный шифр <<Кузнечик>> (ГОСТ Р 34.12-2015). Программная реализация блочного шифра требует выполнения ресурсоемких раундовых преобразований (S-box, линейное преобразование) для каждого блока данных (16 байт);
    \item \textbf{Отсутствие SIMD-оптимизаций:} Оригинальный модуль \texttt{WireGuard} активно использует векторные инструкции процессора (AVX2/AVX-512) для параллельной обработки блоков. Текущая реализация \texttt{WireGost} выполнена на чистом Си без использования ассемблерных вставок, что оставляет значительный потенциал для оптимизации в будущем.
\end{itemize}

Тем не менее, достигнутая скорость в $\approx 750$ Мбит/с является достаточной для насыщения большинства каналов связи ($100$ Мбит/с и приближается к $1$ Гбит/с), что делает решение пригодным для практического использования.

\nlparagraph{Сравнение \texttt{WireGost} с \texttt{ruWireGuard}}

Сравнение с \texttt{ruWireGuard} наглядно демонстрирует неэффективность реализации высоконагруженных VPN-решений в пространстве пользователя (\texttt{userspace}) и использования языка программирования Go для тяжелых алгоритмов.
\begin{itemize}
    \item \textbf{Пропускная способность:} перенос кода в ядро позволил увеличить скорость в \textbf{12 раз} (с 61 до 743 Мбит/с);
    \item \textbf{Стабильность UDP:} реализация на \texttt{Go} теряет более 98\% \texttt{UDP}-пакетов при нагрузочном тестировании. Это происходит из-за того, что планировщик ОС и сборщик мусора (\texttt{GC}) языка \texttt{Go} не успевают обрабатывать прерывания и переключать контекст (\texttt{Kernel} $\leftrightarrow$ \texttt{User}) с необходимой скоростью. Модуль ядра \texttt{WireGost} демонстрирует потери на уровне 13\%, что является приемлемым показателем при полной загрузке \texttt{CPU};
    \item \textbf{Задержки:} задержка в \texttt{ruWireGuard} (441 мс) делает его непригодным для VoIP и интерактивных приложений. \texttt{WireGost} обеспечивает задержку менее 0.1 мс.
\end{itemize}

\nlparagraph{Парадокс низкой задержки}

Интересно отметить, что задержка под нагрузкой (\texttt{Latency}) у \texttt{WireGost} (0.083 мс) оказалась даже ниже, чем у оригинального \texttt{WireGuard} (0.098 мс). Это явление возможно объясняется эффектом \textit{Bufferbloat}. Из-за сверхвысокой пропускной способности оригинального \texttt{WireGuard} сетевые буферы (\texttt{qdisc}) заполняются максимально плотно, создавая очереди. В случае с \texttt{WireGost} «узким местом» является \texttt{CPU}, а не сетевая очередь, поэтому пакеты, успевшие зашифроваться, отправляются в сеть практически мгновенно, не ожидая в переполненном буфере.\\

С другой стороны, если данное предположение ошибочно, то отличия в задержке являются незначительными и их можно списать на погрешность измерений.

\subsection{Выводы по главе}

В ходе выполнения раздела автором была решена задача программной реализации протокола \texttt{WireGost} в виде загружаемого модуля ядра Linux.

\begin{enumerate}
    \item разработаны и интегрированы в подсистему \texttt{Crypto API} модули российских криптографических алгоритмов: хэш-функция <<Стрибог>> (с поддержкой \texttt{HMAC}), блочный шифр <<Кузнечик>> (в режиме \texttt{MGM}) и протокол выработки общего ключа \texttt{VKO} ГОСТ Р 34.10-2012;
    \item модифицирована машина состояний протокола \texttt{Noise} и механизм рукопожатия для поддержки увеличенных размеров ключей (512 бит);
    \item адаптирован интерфейс \texttt{Netlink} и разработана утилита \texttt{wg\_gost\_cli} для управления интерфейсом, так как стандартные утилиты несовместимы с новыми размерами ключей;
    \item проведенное нагрузочное тестирование показало, что разработанное решение обеспечивает максимальную пропускную способность (по \texttt{UDP}) до \textbf{862 Мбит/с}, что в \textbf{20 раз} превосходит существующие аналоги в пространстве пользователя, обеспечивая при этом надежную защиту данных в соответствии с отечественными стандартами.
\end{enumerate}

Подтверждена корректность работы всех криптографических примитивов и протокола, в целом (тесты целостности файлов и успешное прохождение рукопожатий). В случае дальнейшей доработки разработанный модуль возможен для эксплуатации в средах с высокими требованиями к производительности.
