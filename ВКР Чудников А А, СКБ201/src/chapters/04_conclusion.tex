\section{Заключение}

В ходе выполнения выпускной квалификационной работы было проведено исследование методов реализации виртуальных сетевых интерфейсов в ядре операционной системы \texttt{Linux} и успешно решена задача создания высокопроизводительного \texttt{VPN}-решения, использующего отечественные криптографические стандарты.

\subsection{Текущие результаты}

В рамках работы были достигнуты следующие результаты:

\begin{enumerate}
    \item \textbf{Проведен анализ архитектуры ядра Linux:} изучены механизмы работы сетевого стека, подсистемы \texttt{Crypto API} и протокола \texttt{Netlink}. Обоснован выбор реализации в пространстве ядра (\texttt{Kernel Space}) как единственного способа обеспечения высокой производительности для высоконагруженных каналов связи;

    \item \textbf{Разработан комплекс криптографических модулей:} реализованы и внедрены в ядро \texttt{Linux} алгоритмы ГОСТ Р 34.11-2012 (<<Стрибог>>), ГОСТ Р 34.12-2015 (<<Кузнечик>>) в режиме \texttt{MGM} и протокол выработки общего ключа \texttt{VKO} ГОСТ Р 34.10-2012. Все криптографические алгоритмы реализованы в виде модулей реализующих интерфейсы \texttt{Linux Crypto API};

    \item \textbf{Создан прототип VPN-интерфейса WireGost:} на базе архитектуры протокола \texttt{WireGuard} и фреймворка \texttt{Noise Protocol Framework} разработан модуль ядра, обеспечивающий построение защищенных туннелей с использованием отечественных криптографических алгоритмов;

    \item \textbf{Разработан инструментарий управления:} создана утилита конфигурации \texttt{wg\_gost\_cli}, взаимодействующая с ядром через протокол \texttt{Generic Netlink}, что позволило обойти ограничения стандартных утилит \texttt{WireGuard};

    \item \textbf{Проведено нагрузочное тестирование:} экспериментально подтверждено, что перенос реализации ГОСТ-криптографии в пространство ядра обеспечивает существенный прирост производительности. Пропускная способность разработанного решения достигает в случае \texttt{UDP} \textbf{862 Мбит/с} и в случае \texttt{TCP} -- \textbf{765 Мбит/с}, что в \textbf{20 раз} и \textbf{12 раз} соответственно превосходит существующие аналоги, работающие в пространстве пользователя (\texttt{ruWireGuard-Go}).
\end{enumerate}

Сравнительный анализ показал, что, несмотря на отставание от оригинального \texttt{WireGuard} (обусловленное высокой вычислительной сложностью блочного шифра <<Кузнечик>> по сравнению с потоковым \texttt{ChaCha20}), разработанный модуль \texttt{WireGost} пригоден для практического использования в сетях с пропускной способностью до 1 Гбит/с.

\subsection{Дальнейшее направление развития проекта}

Основным направлением развития проекта является оптимизация криптографических примитивов с использованием векторных инструкций процессора (AVX/SSE) для повышения скорости шифрования, а также улучшение поддержки отечественной криптографии в ядре операционной системы \texttt{Linux}.\\

Более того, требуется дальнейшее исследование и разработка инструмента для конфигурации и управления модулем \texttt{WireGost}, так как текущий интерфейс не приспособлен для массового использования.\\

Разработанное программное обеспечение может служить основой для создания доверенных отечественных средств защиты информации, интегрируемых непосредственно в ядро операционной системы \texttt{Linux}.
