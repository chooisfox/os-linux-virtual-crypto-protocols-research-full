\unnumsection{Введение}

Выпускная квалификационная работа посвящена исследованию и практической реализации виртуального криптографического сетевого интерфейса в пространстве ядра операционной системы \texttt{Linux}, с использованием отечественных криптографических стандартов.

\unnumsubsection{Актуальность темы исследования}

Распространение практики использования криптографических стандартов, разработанных и стандартизированных на территории недружественных нашему государству стран, является одной из главных проблем отечественной криптографии.
Данная проблема становится наиболее актуальной в связи с геополитическими изменениями, происходящими в мире на данный момент.
Использование таких стандартов влечет за собой реальные риски для безопасности государственных и коммерческих систем, так как никто не может гарантировать, что в них не были заложены как программные закладки, так и не задокументированные возможности, как, например, в генераторе псевдослучайных чисел Dual EC.~\cite{dual_ec_backdoor}
Единственной надежной защитой от такого рода угроз становится повсеместный переход на отечественные криптографические стандарты.\\

Однако, в таком случае появляется другая проблема: практически полное отсутствие каких-либо отечественных аналогов довольно большого пласта ПО, от которого зависят как коммерческие, так и государственные системы.
Данная работа в первую очередь стремится к поиску методов решения такого технического отставания в сфере ПО для поднятия виртуальных частных сетей.
В работе рассматривается возможность реализации такого протокола и рассматриваются проблемы, связанные с разработкой. В качестве основы был выбран зарубежный протокол \texttt{WireGuard}, так как его реализация значительно проще и быстрее его прямых аналогов. Более того, отечественные стандарты обладают аналогами всех криптографических алгоритмов, используемых в данном протоколе.\\

\unnumsubsection{Научная новизна}
Предпринимаемые ранее попытки переноса \texttt{WireGuard} на отечественные криптографические стандарты~\cite{bi_zone_wireguard} основывались на идее реализации \texttt{userspace}-версии протокола, которая является значительно более медленной. В данной работе предлагается реализация \texttt{kernel}-версии протокола, которая должна быть значительно быстрее и эффективнее.

\unnumsubsection{Объект исследования}

Объектом исследования в данной работе выступают ядро операционной системы \texttt{Linux}, сетевой стек ядра, криптографический API (\texttt{Crypto API}), а также механизмы передачи информации из пространства пользователя в ядро, такие как протокол <<\texttt{Netlink}>>.

\unnumsubsection{Предмет исследования}

Предметом исследования выступают конкретные механизмы, архитектурные решения и программные интерфейсы, используемые в \texttt{WireGuard}, а также методы интеграции в нее отечественных криптографических протоколов.

\unnumsubsection{Степень научной разработанности темы}
Наиболее глубоко исследованной частью данной работы являются вопросы касающиеся реализации загружаемых модулей ядра Linux, а также его криптографической подсистемы (\texttt{Crypto API}). Данные вопросы подробно описаны как в технической документации ядра Linux, так и в многочисленных статьях и публикациях. ~\cite{crypto_api_doc, netlink_doc, kernel_modules_doc}\\

Архитектура \texttt{WireGuard} и используемые им криптографические протоколы, подробно описаны его создателем Джейсоном Доненфилдом~\cite{wireguard}, а также отражены в работе <<Analysis of the WireGuard
protocol>> Питера Ву.~\cite{wireguard_analysis}\\

Сама идея переноса протокола \texttt{WireGuard} на криптографические алгоритмы, стандартизированные в ГОСТ (Государственный стандарт), также не является новой: схожее исследование с неутешительными результатами было проведено компанией <<\texttt{BI.ZONE}>> в рамках проекта <<RuWireGuard>>~\cite{bi_zone_wireguard}. Отличительной особенностью данного проекта стала его невероятно медленная реализация, которая была выполнена на языке <<\texttt{Go}>> и функционировала в пространстве пользователя (\texttt{userspace}), что значительно снижало её производительность

\unnumsubsection{Цель дипломной работы}

Целью дипломной работы является исследование возможности встраивания в ядро Linux виртуального криптоинтерфейса, поддерживающего отечественные криптографические алгоритмы.

\unnumsubsection{Методологическая основа исследования}

Аналитическая часть дипломной работы представляет собой системный анализ, изучение и обобщение информации, касающейся затрагиваемых данной работой тем. Анализируемая информация бралась из технической документации, спецификаций протоколов, исследовательских работ, а также из результатов анализа исходного кода программных решений, упомянутых в аналитической части.\\

Практическая часть дипломной работы представляет собой разработку прототипов различных криптографических модулей ядра Linux, а также виртуального криптоинтерфейса на основе архитектуры \texttt{WireGuard}. Сами модули были разработаны с использованием языка программирования <<\texttt{C}>>.\\

Исследовательская часть дипломной работы представляет собой приведенный ниже алгоритм, по которому производился сравнительный анализ интересующих нас решений:
\begin{enumerate}
    \item \textbf{Создание среды тестирования}: создание виртуальной сети с использованием Linux Network Namespaces (\texttt{netns}) и виртуальных пар Ethernet (\texttt{veth});
    \item \textbf{Измерение базовых показателей канала}: измерение характеристик <<голого>> канала без шифрования и туннелирования. С помощью утилит \textit{iperf3} и \textit{ping} замеряется максимальная пропускная способность (TCP/UDP), задержка (RTT) и базовое потребление ресурсов ЦП, которые принимаются за теоретический максимум;
    \item \textbf{Тестирование существующих решений}: проведение измерений для оригинального протокола \texttt{WireGuard} (\texttt{ChaCha20-Poly1305}) и \texttt{userspace}-реализации протокола с отечественной криптографией (\texttt{RuWireGuard} на \texttt{Go});
    \item \textbf{Развертывание разработанного прототипа}: загрузка разработанного модуля ядра \texttt{Linux} (\texttt{WireGost}), настройка виртуальных интерфейсов и установление защищенного соединения с использованием алгоритмов ГОСТ;
    \item \textbf{Комплексное нагрузочное тестирование}: измерение пропускной способности, джиттера и нагрузки на процессор при прохождении трафика через разработанный туннель.
    \item \textbf{Сравнительный анализ}: сопоставление полученных результатов с показателями \texttt{userspace}-решения и оригинального WireGuard. Оценка накладных расходов на шифрование ГОСТ и расчет прироста производительности, достигнутого за счет переноса реализации в пространство ядра.
\end{enumerate}

\unnumsubsection{Структура работы}

Дипломная работа состоит из введения, трех глав, заключения и списка использованных источников:
\begin{enumerate}
    \item \textbf{Введение}: описание целей, задач и актуальности работы. Разбор структуры дипломной работы;
    \item \textbf{Теоретические основы}: обзор базовых определений, сетевых и криптографических подсистем ядра \texttt{Linux}, а также технологий и протоколов, затронутых в работе;
    \item \textbf{Архитектура ядра и WireGuard}: анализ архитектуры ядра \texttt{OS Linux}, архитектуры \texttt{WireGuard} и существующих решений на базе отечественных стандартов;
    \item \textbf{Разработка и экспериментальная оценка}: описание процесса разработки прототипа и результаты экспериментальной оценки его производительности;
    \item \textbf{Заключение}: в заключении подводятся итоги проделанной работы и формулируются основные выводы.
\end{enumerate}
